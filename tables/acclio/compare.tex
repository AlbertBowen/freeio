% \renewcommand{\arraystretch}{2.0}
% \begin{table}[t]
%     \scriptsize
%     \centering
%     \resizebox{\columnwidth}{!}{
%     \begin{tabular}{c|c|c|c}       
%         \specialrule{1.5pt}{0pt}{0pt}
%         & Reactive Rate Control & Fixed Buffer & \sys \\
%         %\cline{1-7}
%         \midrule
%         Limitation         & slow response & packet loss & slow path \\
%         \hline
%         Result      & \makecell{unavoidable \\ LLC miss} & \makecell{unexpectedly\\ trigger CCA} & c1c2 \\       
%         \hline
%         Performance     & \makecell{I/O process\\degradation} & \makecell{network rate \\ degradation} & \makecell{line-rate\\ after design} \\
%         \specialrule{1.5pt}{0pt}{0pt}
%     \end{tabular}}
%     \caption{Comparison of \sys with traditional roadmaps.}
%     \label{tab:compare}
%     % \vspace{-1.5em}
% \end{table}
\begin{table*}[t]
    \footnotesize
    \centering
    \begin{tabular}{c|c|c}
        \toprule
        & {\small \textbf{Limitations}} & {\small \textbf{I/O Performance}} \\
        \midrule
        \makecell{\textbf{Reactive Rate Control}\\(represented as HostCC)} & 
        \makecell{The \myred{slow response} limitation incurs unavoidable LLC misses} & 
        \makecell{Slow I/O path processing rate} \\
        \midrule
        \makecell{\textbf{Fixed Buffer}\\(represented as ShRing)} & 
        \makecell{Unexpectedly trigger CCA to avoid \myred{packet loss}} & 
        \makecell{Slow network transmission rate} \\
        \midrule
        \textbf{\sys} & 
        \makecell{Introduce \textbf{proactive rate control} and \textbf{elastic buffer} to address previous limitations} & 
        \makecell{\textbf{\myred{Near line-rate}} performance} \\
        \bottomrule
    \end{tabular}
    \caption{Comparison of different I/O optimization approaches}
    \label{tab:compare}
    % \vspace{-0.1in}
\end{table*}